% --------------------------------------------------------------
  % This is all preamble stuff that you don't have to worry about.
% Head down to where it says "Start here"
% --------------------------------------------------------------

\documentclass[12pt]{book}

\usepackage[margin=1in]{geometry}
\usepackage{amsmath,amsthm,amssymb,hyperref,blkarray}

\newcommand{\N}{\mathbb{N}}
\newcommand{\Z}{\mathbb{Z}}
\newcommand{\R}{\mathbb{R}}
\newcommand{\Q}{\mathbb{Q}}

\hypersetup{
bookmarks=true,
unicode=true,
pdftitle={Jonathan Bown}
}


\newenvironment{theorem}[2][Theorem]{\begin{trivlist}
\item[\hskip \labelsep {\bfseries #1}\hskip \labelsep {\bfseries #2.}]}{\end{trivlist}}
\newenvironment{lemma}[2][Lemma]{\begin{trivlist}
\item[\hskip \labelsep {\bfseries #1}\hskip \labelsep {\bfseries #2.}]}{\end{trivlist}}
\newenvironment{exercise}[2][Exercise]{\begin{trivlist}
\item[\hskip \labelsep {\bfseries #1}\hskip \labelsep {\bfseries #2.}]}{\end{trivlist}}
\newenvironment{problem}[2][Problem]{\begin{trivlist}
\item[\hskip \labelsep {\bfseries #1}\hskip \labelsep {\bfseries #2.}]}{\end{trivlist}}
\newenvironment{question}[2][Question]{\begin{trivlist}
\item[\hskip \labelsep {\bfseries #1}\hskip \labelsep {\bfseries #2.}]}{\end{trivlist}}
\newenvironment{solution}
               {\let\oldqedsymbol=\qedsymbol
                \renewcommand{\qedsymbol}{$\square$}
                \begin{proof}[\bfseries\upshape Solution]}
               {\end{proof}
                \renewcommand{\qedsymbol}{\oldqedsymbol}}


\newcommand{\prb}[1]{\textbf{Exercise #1.}}

\newenvironment{corollary}[2][Corollary]{\begin{trivlist}
\item[\hskip \labelsep {\bfseries #1}\hskip \labelsep {\bfseries #2.}]}{\end{trivlist}}

\begin{document}

% --------------------------------------------------------------
%                         Start here
% --------------------------------------------------------------

\title{Solutions Manual for Gregory F. Lawler's \\ \emph{Introduction to Stochastic Processes}}%replace X with the appropriate number
\author{Jonathan Bown}
\date{}

\maketitle

\part{}

\chapter{Finite Markov Chains}



%-------------
%      Notation for Exercises
%-------------
\begin{problem}{1.1}
\end{problem}


\begin{problem}{1.2}
Consider a Markov chain with state space {0,1} and transition matrix\\
\begin{center}
$\bf{P} = \begin{bmatrix}
    1/3 & 2/3 \\
    3/4 & 1/4 
\end{bmatrix}$
\end{center}
Assuming that the chain starts in state $0$ at time $n = 0$, what is the probability that it is in state 1 at time $n=3$?
\end{problem}
\begin{solution}
This is just some basic matrix multiplication. The chain starts in state 0 at time $n = 0$ so we will look at the first row of the matrix $\bf{P}^3$. 
\end{solution}


\begin{problem}{1.3}
\end{problem}
\begin{solution}
\end{solution}

\begin{problem}{1.4}
\end{problem}
\begin{solution}
\end{solution}



\begin{problem}{1.5}
\end{problem}
\begin{solution}
(1) Recurrent classes: $\{0,1\}, \{2,4\}$. Transient class: $\{3,5\}$ \\
(2) To analyze large time behavior of the Markov chain on the class $R_{1} = \{0,1\}$, we need only to consider its matrix 
\begin{center}
$ \bf{P}_{\{0,1\}} = 
    \bordermatrix{ & 0 & 1 \cr
      0 & 0.5 & 0.5 \cr
      1 & 0.3 & 0.7 } \qquad
      $
\end{center}
Solving $\pi \bf{P}_{\{0,1\}} = \pi$ we get its invariant probability $\pi = (\frac{3}{8}, \frac{5}{8})$. Then $lim_{n \to \infty} P_{n}(0,0) = \frac{3}{8}$. \\

(3) To find $lim_{n \to \infty} P_{n}(5,0)$, we first find $lim_{n \to \infty} P_{n}(0,R_{1})$, the probability that the chain will be absorbed into $R_{1} = \{0,1\}$. Rearrange $P$ we can write it as 
\begin{center}
$ \bf{\tilde{P}}_{\{0,1\}} = 
    \bordermatrix{ & \{0,1\} & \{2,4\} & 3 & 5 \cr
      \{0,1\} & 1 & 0 & 0 & 0 \cr
      \{2,4\} & 0 & 1 & 0 & 0 \cr
      3 & 0.5 & 0.25 & 0 & 0.25 \cr
      5 & 0.2 & 0.2 & 0.2 & 0.4}  = \begin{bmatrix}
    I & 0 \\
    S & Q 
\end{bmatrix} $
      

      
\end{center}
Then it follows from $lim_{n \to \infty} \bf{\tilde{P}}^{n} = \begin{pmatrix} I & 0 \\ (I - Q)^{-1}S & 0 \end{pmatrix}$ (see Section 1.5) and 
\begin{center}
$(I - Q)^{-1}S = \frac{1}{11} * \begin{pmatrix} 7 & 4 \\ 6 & 5 \end{pmatrix}$ 
\end{center}
that $lim_{n \to \infty} P_{n}(5, R_{1}) = \frac{6}{11}$. Combining it with (2) we get $lim_{n \to \infty} P_{n}(5,0) = \frac{6}{11}* \frac{3}{8} = \frac{9}{44} = .2045$
\end{solution}


\begin{problem}{1.6}
\end{problem}
\begin{solution}
\end{solution}



\begin{problem}{1.7}
\end{problem}
\begin{solution}
\end{solution}


\begin{problem}{1.8}
\end{problem}
\begin{solution}
\end{solution}

\begin{problem}{1.9}
\end{problem}
\begin{solution}
\end{solution}


\begin{problem}{1.10}
\end{problem}
\begin{solution}
\end{solution}

\begin{problem}{1.11}
\end{problem}
\begin{solution}
\end{solution}


\begin{problem}{1.12}
\end{problem}
\begin{solution}
\end{solution}


\begin{problem}{1.13}
\end{problem}
\begin{solution}
\end{solution}


\begin{problem}{1.14}
\end{problem}
\begin{solution}
\end{solution}


\begin{problem}{1.15}
\end{problem}
\begin{solution}
\end{solution}


\begin{problem}{1.16}
\end{problem}
\begin{solution}
\end{solution}


\begin{problem}{1.17}
\end{problem}
\begin{solution}
\end{solution}


\begin{problem}{1.18}
\end{problem}
\begin{solution}
\end{solution}


\begin{problem}{1.19}
\end{problem}
\begin{solution}
\end{solution}


\begin{problem}{1.20}
\end{problem}
\begin{solution}
\end{solution}


\begin{problem}{1.21}
\end{problem}
\begin{solution}
\end{solution}



\chapter{Countable Markov Chains}

\begin{problem}{2.1}
\end{problem}
\begin{solution}
\end{solution}



\chapter{Continuous-Time Markov Chains}

\begin{problem}{3.1}
\end{problem}
\begin{solution}
\end{solution}




\chapter{Optimal Stopping}

\begin{problem}{4.1}
\end{problem}
\begin{solution}
\end{solution}





\chapter{Martingales}

\begin{problem}{5.1}
\end{problem}
\begin{solution}
\end{solution}



\chapter{Renewal Processes}

\begin{problem}{6.1}
\end{problem}
\begin{solution}
\end{solution}




\chapter{Reversible Markov Chains}

\begin{problem}{7.1}
\end{problem}
\begin{solution}
\end{solution}



\chapter{Brownian Motion}

\begin{problem}{8.1}
\end{problem}
\begin{solution}
\end{solution}





\chapter{Stochastic Integration}


% --------------------------------------------------------------
%     You don't have to mess with anything below this line.
% --------------------------------------------------------------

\end{document}